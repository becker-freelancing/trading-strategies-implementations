\section{Introduction}

\subsection{Motivation}

In recent years, the cryptocurrency market has emerged as a highly dynamic and rapidly evolving financial ecosystem.
Trading pairs such as ETH/USDC at minute-level (M1) intervals provide vast amounts of high-frequency data, reflecting pronounced volatility and frequent market shifts.
This environment presents both significant challenges and promising opportunities for traders and researchers.

The availability of detailed tick and minute data, combined with direct API access from cryptocurrency brokers, enables the development of sophisticated data-driven trading systems.
Advances in machine learning and deep learning offer powerful tools for identifying patterns in market movements.

This motivates the exploration of AI-based trading systems specifically designed for the ETH/USDC pair, aiming to leverage technical features and advanced models to enhance performance in the volatile cryptocurrency environment.
At the same time, classical trading strategies could also perform well in this setting, particularly when adapted to specific market regimes.

Techniques such as moving averages, Bollinger Bands, and momentum indicators offer the advantages of simplicity and transparency, allowing traders to better understand and trust their decision-making processes.


The ETH/USDC trading pair at the minute timescale was deliberately chosen for this research because it offers several key advantages for the analysis and modeling of algorithmic trading strategies.
Ethereum is one of the largest and most liquid cryptocurrencies worldwide, while USDC, as a stablecoin, experiences relatively low price fluctuations.
This combination provides a clear separation between the volatile asset (ETH) and a stable reference currency (USDC), simplifying analysis.

The high liquidity of this pair further ensures a reliable data foundation with minimal slippage and tight spreads.
Moreover, the M1 timescale offers sufficiently granular information to capture short-term market movements and micro-patterns without relying exclusively on high-frequency tick data.
This creates a practical and balanced framework for the development and evaluation of data-driven trading algorithms.


\subsection{Aim of this Paper}


The primary goal of this paper is to develop and evaluate both AI-powered and classical trading strategies for the ETH/USDC cryptocurrency pair.

Specifically, the work will:


\begin{enumerate}

    \item Retrieve historical ETH/USDC data via a broker API and preprocess it for analysis.

    \item Perform exploratory data analysis and feature engineering, incorporating trend, volatility, and momentum indicators.

    \item Identify and classify distinct market regimes to enable adaptive trading decisions.

    \item Apply risk and money management techniques to realistically simulate trading performance.

    \item Design, train, and benchmark multiple deep learning architectures, including LSTM, CNN, and hybrid models, for price forecasting and trade decision classification.

    \item Develop algorithmic trading strategies based on AI model predictions and compare them with classical technical trading approaches.

    \item Implement a modular trading engine capable of backtesting strategies and executing trades in real time through broker integration.

    \item Deploy the most promising trading strategy in a controlled live test on a demo account with a real broker.

\end{enumerate}

\noindent
By focusing on the ETH/USDC M1 pair, this paper aims to generate insights into the comparative effectiveness of deep learning–based and classical strategies in cryptocurrency trading, while delivering practical tools for automated and adaptive trading in this challenging asset class.


\subsection{Structure of this Paper}


\autoref{chap:broker} lays the foundation for the technical integration with financial markets.
This includes selecting a suitable broker and setting up an automated interface for data retrieval.
Without this foundation, real-time access to market data and the subsequent live execution of strategies would not be possible.

\autoref{chap:market-regimes} and \autoref{chap:risk-man} introduce essential concepts such as market regimes as well as money and risk management.
These theoretical foundations enable a structured analysis of the markets while ensuring that trading decisions are made within a solid risk management framework.
Classification approaches for market phases, along with rules for position sizing and risk limitation, are covered.

\autoref{chap:performance} then introduces methods and metrics for comparing trading strategies.
These include equity curves, maximum drawdown, maximum gain and loss, standard deviations of profits, and win rate.

Building on this, \autoref{chap:eda} presents a comprehensive exploratory data analysis (EDA) approach, outlining key steps for processing and transforming historical price data.
In addition to generating technical indicators, modern techniques such as feature scaling and principal component analysis (PCA) are discussed as a basis for machine learning applications.

In \autoref{chap:dl-models}, various deep learning models are developed and trained to predict future market movements.
Both regression and classification models are examined and evaluated with respect to their suitability for use in trading.

\autoref{chap:trading-strategies} focuses on the development of concrete trading strategies.
These are based either on classic technical indicators or on signals derived from the previously developed ML models.
Each strategy is described in detail and prepared for subsequent evaluation.

To execute and evaluate these strategies, \autoref{chap:te} introduces a modular trading engine that supports both backtesting and live trading.
Its plug-in architecture allows for the flexible integration of different strategies.

\autoref{chap:backtesting} assesses the performance of the developed strategies using backtests.
Various metrics such as profit curves, drawdowns, and win rates are considered, along with the impact of trading fees, to ensure a realistic evaluation.

\autoref{chap:live-test} describes the transition to practical application through a live test, demonstrating the extent to which backtest results can be confirmed in real market conditions.
Finally, \autoref{chap:overfitting} revisits the issue of overfitting in trading strategies.

\autoref{chap:conclusion} concludes the thesis with a summary of the main findings, while \autoref{chap:aim-further-works} provides suggestions for future work, such as improving the models, refining strategies, or extending the trading engine.

\subsection{Ethereum}

Ethereum is a decentralized, open-source blockchain platform introduced in 2015 by Vitalik Buterin and other co-founders.
Unlike Bitcoin, which was primarily designed as a digital currency, Ethereum is a Turing-complete platform capable of running smart contracts and decentralized applications (dApps).
Its native cryptocurrency is called Ether (ETH) \cite{eth-1}.


The Ethereum blockchain is a distributed and immutable ledger that stores all transactions in chronological order.
Each block contains a set of transactions linked together through cryptographic methods.
The integrity and security of the chain are maintained by the Ethereum Virtual Machine (EVM), which executes smart contracts identically across all full nodes \cite{eth-1}.


Unlike centralized systems where a single entity holds control, blockchain technology enables trustless consensus among network participants.
This makes Ethereum particularly relevant for applications in decentralized finance (DeFi), digital identity systems, and asset tokenization \cite{eth-1}.


Ethereum offers both technical opportunities and limitations that directly affect financial applications, DeFi protocols, and automated trading strategies.
One of its key strengths lies in decentralization and transparency.
All transactions and contract states are publicly verifiable and resistant to censorship or manipulation.
In addition, Ethereum’s programmability through smart contracts enables the encoding of complex financial logic, process automation, and a reduced reliance on intermediaries \cite{eth-3}.


However, these strengths are accompanied by notable limitations.
Ethereum continues to face scalability challenges.
Despite recent upgrades, its transaction throughput remains limited compared to centralized systems, leading to potential congestion and elevated fees during periods of high usage.
This is compounded by volatile transaction fees, which make small or frequent transactions economically inefficient.
Furthermore, regulatory uncertainty persists, as Ethereum’s legal classification remains unresolved in many jurisdictions, posing challenges particularly for institutional participants \cite{eth-3}.


In summary, Ethereum provides a powerful and flexible foundation for building decentralized financial systems.
Its capability to execute programmable logic on-chain without central authorities makes it an ideal infrastructure for digital assets.
Nevertheless, limitations in scalability, cost, and regulatory clarity must be addressed when designing systems that depend on Ethereum.


\subsection{USD Coin}


In addition to traditional cryptocurrencies such as Bitcoin or Ethereum, a special category of digital assets known as \textit{stablecoins} has emerged.
One of the best-known stablecoins is the USD Coin (USDC), whose value is pegged to the US dollar (USD).

USDC is a digital stablecoin that operates on multiple blockchains, including Ethereum, Solana, and Polygon.
It was launched in 2018 by the company Circle in collaboration with the cryptocurrency platform Coinbase.
The primary objective of USDC is to digitally represent the value of one US dollar, meaning that one USDC token is intended to correspond to the equivalent of one USD \cite{usdc-1}.


Unlike volatile cryptocurrencies, USDC is backed by a fiat currency.
To maintain its peg, for every USDC issued, a real US dollar, or an equivalent liquid reserve instrument, is held in custody with regulated financial institutions.
The existence and adequacy of these reserves are regularly audited by independent auditors, with the results made publicly available \cite{usdc-2}.


USDC is fully programmable and can be transferred globally in near real time without the need for traditional financial intermediaries such as banks.
This makes it particularly attractive for applications in decentralized finance (DeFi), cryptocurrency trading, and international remittances \cite{circle-usdc}.


Although USDC aims to maintain a 1:1 peg with the US dollar, the two are not the same.
The US dollar (USD) is the official fiat currency of the United States, issued under the authority of the US Federal Reserve.
It exists both in physical form (cash) and digitally (e.g., in bank accounts) and is recognized as legal tender \cite{usdc-2}.


USDC, by contrast, is a privately issued product of a technology company.
It is a tokenized representation of the US dollar that exists exclusively in the digital domain.
USDC is not legal tender and is not directly controlled by a central bank.
Its peg to the US dollar depends on trust in the issuer’s ability to maintain adequate reserves.


