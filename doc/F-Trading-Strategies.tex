\section{Trading-Strategies}

In the context of automated trading, systematic trading strategies play a central role.
They enable decisions to be made based on clearly defined rules or mathematical models, rather than on subjective assessments or human intuition.

A trading strategy defines a methodical approach by which financial instruments are bought or sold to achieve a specific goal.
Typically, this is to maximize profit while limiting risk.
Such strategies are often based on indicators from technical analysis or fundamental market information \cite{investopia-trading-strategy}.

Essential components of a strategy are signal generation (entry/exit criteria), risk management (e.g., stop-loss, position sizes), and performance evaluation based on metrics such as cumulative profit, volatility, or maximum drawdown.
The development process of a successful strategy typically comprises several phases, ranging from idea generation and backtesting to optimization and validation on previously unseen market data \cite{investopia-trading-strategy-components}.

This chapter provides an overview of the trading strategies compared in this paper.
Both the models trained in \autoref{chap:dl-models} and classic trading strategies are presented.
An overview of the metrics used to compare the trading strategies is also provided.

%TODO: Es gibt parameter die permutiert werden -> Min, Max, Step erklären
% TODO: PositionBehaviour ist immer ein parameter
\subsection{AI-Trading-Strategies}

AI trading strategies treat the model outputs differently.
Here, too, a distinction is made between regression and classification models.

\subsubsection{Regression AI-Trading-Strategies}

As described in \autoref{chap:regression-models}, the regression models output a sequence of logarithmic returns for the next 30 minutes.
From this, the expected price in $i$ minutes can be calculated based on the current price $P_t$.

\[
    Price_{t+i} = Price_t * \prod_{k=0}^{i} e^{LogReturn_{t+k}}
\]

Based on the predicted price, the stop-loss and take-profit can be determined identically to the loss-function of the regression models (\autoref{chap:regression-loss}).
This creates an entry signal with fixed stop-loss and take-profit level.

However, since it is possible that the price could exceed the stop level or the take-profit level might not be reached, additional parameters are introduced into the strategy that shift the two levels by a specific number of points.
A third parameter is also introduced, which specifies a fixed distance to the stop loss if no stop loss has been predicted.
This can happen, for example, if the initial prediction is positive and the price in the prediction does not fall below the current price.

\begin{table}[H]
    \centering
    \begin{tabular}{cccc}
        \toprule
        Parameter Name & Min Value & Max Value & Step Size
        \\
        \midrule
        \textbf{Take-Profit Delta}             & -20.0 & 20.0 & 2.0  \\
        \textbf{Stop-Loss Delta}               & -20.0 & 20.0 & 2.0  \\
        \textbf{Stop-Loss not Predicted Delta} & 1.0   & 20.0 & 20.0 \\
        \bottomrule
    \end{tabular}
    \caption{AI Regression Model Strategy Parameters}
    \label{tbl:regression-strategy-parameters}
\end{table}

\subsubsection{Classification AI-Trading-Strategy}

As described in \autoref{chap:classification-models} the classification models predict a probability for an action, which can be either buy, sell or do nothing.
This strategy executes a buy or sell action if the predicted probability is greater than a predefined minimum probability.
If the buy or sell probability is less than the minimum probability or the do nothing probability is the greatest, the strategy does nothing.

The stop-loss and take-profit levels are also predefined parameters.

\begin{table}[H]
    \centering
    \begin{tabular}{cccc}
        \toprule
        Parameter Name & Min Value & Max Value & Step Size
        \\
        \midrule
        \textbf{Take-Profit Distance}       & 5.0 & 100.0 & 5.0 \\
        \textbf{Stop-Loss Distance}         & 5.0 & 100.0 & 5.0 \\
        \textbf{Min. Probability for Entry} & 0.3 & 0.9   & 0.1 \\
        \bottomrule
    \end{tabular}
    \caption{AI Classification Model Strategy Parameters}
    \label{tbl:classification-strategy-parameters}
\end{table}

\subsection{Classic Trading-Strategies}

\subsection{Performance Comparison}

%TODO: Trading-Strategien wurden auf dem Val-Set gefitted
%TODO: Kann ich auch die Parameter auf dem Val-Set permutieren und so das beste aussuchen, weil die Parameter sind ja immer festgesetzt vorab, anders wie beim klassischen ML, die ja durch das Trainset lernen. Somit wäre ja die evaluation der ML Modelle das durchlaufen des Val-Sets mit festen parametern, was dem entspricht, dass die parameter der strategie vorher festgesetzt werden und dann auf dem Valset "getestet"
