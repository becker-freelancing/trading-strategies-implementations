\section{Money- and Risk-Management}

Successful trading is not only based on a good strategy, but also on a disciplined management of capital and risk.
Even the best prediction is useless if losses grow uncontrolled or a majority of the capital is risked by a few wrong decisions.
This is where money and risk management come into play.
They define clear rules regarding how much to invest per trade, what level of risk is acceptable, and how losses can be limited.
The goal is to protect capital over the long term, minimizing drawdowns, and profit from positive expected values in a controlled manner.
This chapter introduces fundamental concepts, metrics, and methods helping to make rational and sustainable decisions.

\subsection{Calculating the Position Size}

An important part of risk management is calculating the position size.
It is helpful to maximize the potential returns, as well as minimizing the financial risk.
Many traders are only willing to risk 1\% or 2\% of the available capital per trade, to prevent a series of losing trades from decimating the available capital too much.

The distance between the estimated entry price and the estimated stop loss price represents the maximum distance a market can move in an unprofitable direction before a position is automatically closed.

This allows the calculation of the position size to be carried out in three steps:

\begin{enumerate}
    \item \textbf{Determining the risk per trade:} First, it must be determined how much of the available capital should be risked.
    If 1\% of \$10,000 is to be risked, the maximum risk is \$100.
    It is important to note that the fraction of 1\% does not have to be fixed.
    Thus, it is possible to risk more if the entry signal is very clear.
    If the entry signal is less clear, it can also be risked less.
    Only a fixed upper limit should be defined.
    \item \textbf{Calculating the risk per unit:} To calculate the risk per share the absolute distance between the estimated entry price, and the estimated stop loss price must be calculated.
    This represents the risk per unit.
    \item \textbf{Calculate the position size:} Dividing the risked capital by the risk per unit represents the number of units to buy or sell.
\end{enumerate}

In total, these three steps can be combined into one formula \cite{britannica-position-size}:

\begin{equation}
    PositionSize = \frac{Available Balance*RiskPerTrade}{RiskPerUnit}
\end{equation}

So if the available account balance is \$10,000.00, the risk per trade is 1\%, and the distance from the estimated entry price to the stop loss is \$5.
The position size is calculated by:

\begin{equation}
    PositionSize = \frac{\$10.000*1\%}{\$5}=\frac{\$100}{\$5}=20 [Units]
\end{equation}

It is important to note that numbers can result with many decimal places, and many brokers only allow positions with a certain number of decimal places.
If this is the case, the position size must be subsequently rounded to the maximum number of decimal places supported.

\subsection{Validating an Entry Signal}

Not every entry signal generated by a trading strategy is necessarily profitable.
To be profitable in the long term, it is important to ensure that entry signals that are too risky or unrealistic in advance are filtered out, thus preventing positions from being opened.
This chapter presents some techniques that can be used to validate entry signals.

\subsubsection{Risk-Reward-Ratio}

The risk reward ratio (RRR) is a fundamental key figure in trading.
It describes the ratio between the potential profit (reward), and the potential loss (risk) of a single trade.
The RRR helps in deciding whether an entry signal is too risky or not.
It ensures that not only the hit rate determines the success of a strategy, but also the ratio of profit to loss in each individual trade.

The RRR is calculated by:

\begin{equation}
    RRR = \frac{PossibleProfit}{PossibleLoss} = \frac{|OpenPrice - TakeProfitPrice|}{|OpenPrice - StopLossPrice|}
\end{equation}

For example, if a long trade is opened at \$100, with a take profit at \$110, and a stop loss at \$95, the result is:

\begin{equation}
    RRR = \frac{|\$100 - \$110|}{|\$100 - 95\$|} = \frac{\$10}{\$5} = 2
\end{equation}

This means that for every dollar risked, a potential profit of two dollars is targeted.

A RRR greater $1$ is generally considered positive because the expected profit is higher than the potential loss.
However, the RRR should not be viewed in isolation.
The essential factor is the combination of RRR, and hit rate:

\begin{enumerate}
    \item \textbf{High RRR, low hit rate:} e.g.
    $RRR=3$ with only a 30\% probability of winning $\Rightarrow$ potentially profitable.
    \item \textbf{Low RRR, high hit rate:} e.g.
    $RRR=0.5$ with an 80\% hit rate $\Rightarrow$ also potentially profitable.
\end{enumerate}

The following rule of thumb clarifies when a strategy has a positive expected value in the long term:

\begin{equation}
    ExpectedValue = PossibleProfit * HitRatio - PossibleLoss * (1 - HitRatio)
\end{equation}

Only when this expected value is above zero, a trading strategy is statistically profitable.

The RRR is not only a mathematical metric, but a central component of risk management.
It helps traders systematically plan how much they are willing to lose per trade, relative to the expected profit.
By consistently applying a minimum RRR (e.g.
$\ge 1.5$), many inefficient setups can be eliminated in advance \cite{bitpanda-crv}.

\subsubsection{Maximum Account Risk}

(Nicht mehr als 10\% des Riskieren in Summe)

\subsubsection{Minimum Take Profit}

In every trading strategy, transaction fees play an important role.
Especially in short-term trading, transaction fees can turn seemingly profitable trades negative if they are not adequately considered.
A common mistake is setting the take profit level too narrowly, resulting in a profit that is smaller than the costs incurred.
To trade profitably and sustainably, it is therefore essential that the take profit at least covers the fees incurred, but ideally, significantly higher.

\subsection{Dealing with Trading Fees}
\label{chap:dealing-with-trading-fees}

As shown in \autoref{tbl:broker-comparision} ByBit charges two different fees named maker-, and taker-fee.
The maker fee is charged when a limit order is placed in the order book, thereby creating liquidity.
In contrast, the taker fee is charged when a market order is executed.
This removes liquidity from the market.
It is better for a broker if a market is as liquid as possible.
Therefore, maker orders incur lower fees than taker orders.

Especially in higher-frequency trading, it is better to charge as few fees as possible.
Therefore, it is better for a trader to execute as many limit orders as possible.
Two orders are required for a complete trade: one for entry, and one for exit.
But typically, three orders are placed (entry, stop-loss, take-profit), with either the take-profit or stop-loss order being executed.

If a market order is to be executed as an entry order, it is possible to convert it into a limit order by setting the order price slightly below (for long positions) or slightly above the current price (for short positions).
The same procedure can be followed for the take-profit order.

However, this procedure should not be used for a stop-loss order.
If this is converted to a limit position, there is no longer any guarantee that the stop order will be executed at all, as there is no longer any guarantee that the price will rise above or below the order level.
This means that a position may remain open significantly longer than intended.
If the price then continues to move in an unprofitable direction, the worst-case scenario is that the position is automatically closed by the broker because there is no longer any capital in the account.

The same problem can theoretically occur with a take-profit order.
The difference here is that the set stop market order still defines the maximum risk.
Thus, while this particular trade may not be profitable, it is impossible to lose all the capital.
It is similar with the opening order.
If it is not executed, the entire trade is not going to be executed.
This means a missed potential profit, but there is no risk of losing capital.

If all orders are executed as planned, and the price moves in a profitable direction, the conversion of the orders will result in a reduction of fees by a factor of 2.75.