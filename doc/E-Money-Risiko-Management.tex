\section{Money- and Risk-Management}
\label{chap:risk-man}

Successful trading is not only based on a good strategy but also on a disciplined management of capital and risk.
Even the best prediction is useless if losses grow uncontrolled or if a majority of the capital is risked by a few wrong decisions.
This is where money and risk management are applied.
They define clear rules regarding how much to invest per trade, what level of risk is acceptable, and how losses can be limited.
The goal is to protect capital over the long term, minimize drawdowns, and profit from positive expected values in a controlled manner.
This chapter introduces fundamental concepts, metrics, and methods helping to make rational and sustainable decisions.

\subsection{Calculating the Position Size}

An important part of risk management is calculating the position size.
It is helpful both to maximize the potential returns and to minimize financial risk.
Many traders are only willing to risk 1\% or 2\% of the available capital per trade to prevent a series of losing trades from decimating the available capital too much \cite{1-2-prozent}.

The distance between the estimated entry price and the estimated stop-loss price represents the maximum distance the market can move in an unprofitable direction before a position is automatically closed.

This allows the calculation of the position size to be carried out in three steps:

\begin{enumerate}
    \item \textbf{Determining the risk per trade:} First, it must be determined how much of the available capital should be risked.
    If 1\% of \$10,000 is to be risked, the maximum risk is \$100.
    It is important to note that the fraction of 1\% does not have to be fixed.
    Thus, it is possible to risk more if the entry signal is very clear.
    If the entry signal is less clear, less can be risked.
    Only a fixed upper limit should be defined.
    \item \textbf{Calculating the risk per unit:} To calculate the risk per share, the absolute distance between the estimated entry price, and the estimated stop-loss price must be calculated.
    This represents the risk per unit.
    \item \textbf{Calculate the position size:} Dividing the risked capital by the risk per unit gives the number of units to buy or sell.
\end{enumerate}

\noindent
In total, these three steps can be combined into one formula \cite{britannica-position-size}:

\[
    \text{Position Size} = \frac{\text{Available Balance} \cdot \text{Risk Per Trade}}{\text{Risk Per Unit}}
\]

\noindent
So if the available account balance is \$10,000.00, the risk per trade is 1\%, and the distance from the estimated entry price to the stop-loss is \$5.
The position size is calculated by:

\[
    \text{Position Size} = \frac{\$10,000 \cdot 1\%}{\$5}=\frac{\$100}{\$5}=20\text{ [Units]}
\]

\noindent
It is important to note that the calculated number of units can result in many decimal places, and many brokers only allow positions with a certain number of decimal places.
If this is the case, the position size must be subsequently rounded to the maximum number of decimal places supported.

\subsection{Validating an Entry Signal}

Not every entry signal generated by a trading strategy is necessarily profitable.
To be profitable in the long term, it is important to ensure that entry signals that are too risky or unrealistic in advance are filtered out, thus preventing positions from being opened.
This section presents some techniques that can be used to validate entry signals.

\subsubsection{Risk-Reward Ratio}

The risk-reward ratio (RRR) is a fundamental metric in trading.
It describes the ratio between the potential profit (reward) and the potential loss (risk) of a single trade.
The RRR helps in deciding whether an entry signal is too risky or not.
It ensures that not only the win rate determines the success of a strategy, but also the ratio of profit to loss in each individual trade \cite{rrr-base}.

The RRR is calculated by:

\[
    RRR = \frac{\text{Possible Profit}}{\text{Possible Loss}} = \frac{|\text{Open Price} - \text{Take-Profit Price}|}{|\text{Open Price} - \text{Stop-Loss Price}|}
\]

\noindent
For example, if a long trade is opened at \$100, with a take-profit at \$110, and a stop-loss at \$95, the result is:

\[
    RRR = \frac{|\$100 - \$110|}{|\$100 - 95\$|} = \frac{\$10}{\$5} = 2
\]

\noindent
This means that for every dollar risked, a potential profit of two dollars is targeted.

A RRR greater $1$ is generally considered positive because the expected profit is higher than the potential loss.
However, the RRR should not be viewed in isolation.
The essential factor is the combination of RRR and win rate:

\begin{enumerate}
    \item \textbf{High RRR, low win rate:} e.g., $RRR=3$ with only a 30\% probability of winning $\Rightarrow$ potentially profitable.
    Here, the profits per successful trade are very large compared to the losses.
    Despite the low hit rate, the high profits can compensate the more frequent losses, making the strategy profitable in the long term.
    \item \textbf{Low RRR, high win rate:} e.g., $RRR=0.5$ with an 80\% win rate $\Rightarrow$ also potentially profitable.
    In this case, the individual wins are smaller than the losses, but the high hit rate ensures that many small wins offset the few larger losses, thus enabling an overall success.
\end{enumerate}

\noindent
The following rule of thumb clarifies when a strategy has a positive expected value in the long term:

\[
    \text{Expected Value} = \text{Possible Profit} \cdot \text{Win Ratio} - \text{Possible Loss} \cdot (1 - \text{Win Ratio})
\]

\noindent
Only when this expected value is above zero, a trading strategy is statistically profitable.

The RRR is not only a mathematical metric, but also a central component of risk management.
It helps traders systematically plan how much they are willing to lose per trade, relative to the expected profit.
By consistently applying a minimum RRR (e.g., $\ge 1.5$), many inefficient setups can be eliminated in advance \cite{bitpanda-crv}.

\subsubsection{Maximum Account Risk}

A central aspect of risk management in algorithmic trading systems is limiting overall risk at the portfolio level.
In this work, a fixed risk threshold of a maximum of 10\% of the total portfolio value has been set.
This threshold serves as an overarching safety limit to prevent excessive risk accumulating through individual positions or simultaneous strategies.

In this context, overall risk refers to the maximum potential loss of all open positions, weighted by their respective stop-loss levels and position sizes.
The 10\% limit means that, aggregated across all positions, the potential loss in the event of a simultaneous stop-out may not exceed one-tenth of the current portfolio value.

The risk limitation serves as a protective mechanism against systemic failures in the trading system, as even in the case of erroneous trades or sudden market turmoil, the total loss remains within a calculable maximum risk.
Therefore, it increases the portfolio resilience against extreme scenarios.

\subsubsection{Minimum Take-Profit}

Transaction fees play a crucial role in every trading strategy.
Especially in short-term trading, transaction fees can turn seemingly profitable trades negative if not adequately accounted for.
A common mistake is setting the take-profit level too narrowly, resulting in a profit that is smaller than the costs incurred.
Profitable and sustainable trading requires that the take-profit covers at least the fees incurred and ideally exceeds them by a considerable amount.

Assuming a trader sets a take-profit target of 0.1\% on a position, while round-trip transaction costs amount to 0.15\%.
Even with a 100\% win rate, the trader would lose money.

\subsection{Dealing with Trading Fees}
\label{chap:dealing-with-trading-fees}

As shown in \autoref{tbl:broker-comparision}, ByBit charges two different types of fees named maker and taker fee.
The maker fee is charged when a limit order is placed in the order book, thereby creating liquidity.
In contrast, the taker fee is charged when a market order is executed.
This removes liquidity from the market.
It is better for a broker if a market is as liquid as possible \cite{liquid-markets}.
Therefore, maker orders incur lower fees than taker orders.

Especially in higher-frequency trading, it is better to pay as few fees as possible.
Therefore, it is better for a trader to execute as many limit orders as possible.
Two orders are required for a complete trade: one for entry, and one for exit.
Typically, however, three orders are placed (entry, stop-loss, take-profit), with either the take-profit or stop-loss order being executed.

If a market order is to be executed as an entry order, it is possible to convert it into a limit order by setting the order price slightly below (for long positions) or slightly above (for short positions) the current price.
The same procedure can be followed for the take-profit order.

However, this procedure should not be used for a stop-loss order.
If converted to a limit order, there is no guarantee the stop will be executed, as the price may never reach the order level.
This can happen when the price jumps above (or below) the order level and the order thus remains in the order book significantly longer.
This means that a position may remain open significantly longer than intended.
If the price then continues to move in an unprofitable direction, the worst-case scenario is that the position is automatically closed by the broker because there is no longer any capital in the account.

The same problem can theoretically occur with a take-profit order.
The difference here is that the set stop market order still defines the maximum risk.
Thus, while this particular trade may not be profitable, it is impossible to lose all the capital.
It is similar with the opening order.
If it is not executed, the entire trade will not be executed.
This means a missed potential profit, but there is no risk of losing capital.

If all orders are executed as planned, and the price moves in a profitable direction, the conversion of the orders will result in a reduction of fees by a factor of 2.75.